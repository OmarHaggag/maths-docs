\documentclass[12pt]{article}
\usepackage{amsmath}
\usepackage{amssymb}
\usepackage{geometry}
\geometry{a4paper, margin=1in}
\usepackage{fancyhdr}
\pagestyle{fancy}
\fancyhead[L]{Pearson Edexcel IAL Statistics 2}
\fancyhead[R]{Chapter 2: Poisson Distributions}
\fancyfoot[C]{\thepage}
\renewcommand{\headrulewidth}{0.4pt}
\renewcommand{\footrulewidth}{0.4pt}
\begin{document}
\centerline{\Large\textbf{Chapter 2: Poisson Distributions}}
\centerline{\textbf{Pearson Edexcel International A Level Statistics 2 Student Book}}

\vspace{0.5cm}
Chapter 2 of the "Pearson Edexcel International A Level Statistics 2 Student Book" is dedicated to \textbf{Poisson Distributions} [1, 2]. This chapter covers various aspects of this probability distribution, including its definition, conditions for its use, how to model real-world situations, its additive property, and its mean and variance [2, 3].

Here are the main ideas discussed in Chapter 2, presented with solved examples from the sources:

\section*{1. The Poisson Distribution Definition and Basic Probability Calculation}

\begin{itemize}
    \item \textbf{Idea}: This section introduces the Poisson distribution, denoted as $X \sim Po(\lambda)$, where $\lambda$ is the average rate of occurrence [4, 5]. It explains that a Poisson distribution is suitable when events occur:
    \begin{itemize}
        \item \textbf{Singly} in space or time [4].
        \item \textbf{Independently} of each other [4].
        \item At a \textbf{constant average rate} within a given interval [4].
    \end{itemize}
    \item \textbf{Formula}: The probability of $x$ occurrences is given by $P(X=x) = \frac{e^{-\lambda}\lambda^x}{x!}$ [5].

    \item \textbf{Solved Example (from Example 1, page 19)} [5]:
    Let a random variable $X$ follow a Poisson distribution $X \sim Po(2.1)$.
    \begin{enumerate}
        \item[\textbf{a}] $P(X=3)$:
        Using the formula $P(X=x) = \frac{e^{-\lambda}\lambda^x}{x!}$ with $x=3$ and $\lambda=2.1$:
        \begin{align*}
        P(X=3) &= \frac{e^{-2.1} (2.1)^3}{3!} \\
        &= \frac{0.12245 \times 9.261}{6} \approx \textbf{0.1890} \quad \text{(4 d.p.)} [5].
        \end{align*}
        \item[\textbf{b}] $P(X > 1)$:
        This is calculated as $P(X > 1) = 1 - P(X \le 1)$ [5].
        From a calculator or Poisson cumulative distribution table, $P(X \le 1) \approx 0.3796$.
        Therefore,
        \begin{align*}
        P(X > 1) &= 1 - 0.3796 = \textbf{0.6204} \quad \text{(4 d.p.)} [5].
        \end{align*}
        \item[\textbf{c}] $P(1 \le X < 4)$:
        This refers to the sum of probabilities for $X=1, X=2,$ and $X=3$ [5].
        \begin{align*}
        P(X=1) &= \frac{e^{-2.1} (2.1)^1}{1!} \approx 0.2571 \\
        P(X=2) &= \frac{e^{-2.1} (2.1)^2}{2!} \approx 0.2700 \\
        P(X=3) &= \frac{e^{-2.1} (2.1)^3}{3!} \approx 0.1890 \\
        P(1 \le X < 4) &= P(X=1) + P(X=2) + P(X=3) \\
        &= 0.2571 + 0.2700 + 0.1890 = \textbf{0.7161} \quad \text{(4 d.p.)} [5].
        \end{align*}
    \end{enumerate}
\end{itemize}

\section*{2. Modelling with the Poisson Distribution (Real-World Applications)}

\begin{itemize}
    \item \textbf{Idea}: This section focuses on identifying real-world scenarios where the Poisson distribution can be used to model the number of times a particular event occurs within a fixed interval of time or space [6]. It emphasizes checking the underlying assumptions of singleness, independence, and constant average rate [6].

    \item \textbf{Solved Example (from Example 4, page 22)} [7]:
    An internet service provider observes that, on average, 4 users every hour fail to connect on their first attempt.
    \begin{enumerate}
        \item[\textbf{a}] Give two reasons why a Poisson distribution might be a suitable model for the number of failed connections every hour.
        \begin{itemize}
            \item Failed connections occur \textbf{singly} (one at a time) and randomly in a given timeframe [7].
            \item The average rate of failed connections is \textbf{constant} (4 per hour) [7].
            \item The occurrences of failed connections are \textbf{independent} [7].
        \end{itemize}
        \item[\textbf{b}] Find the probability that in a randomly chosen hour:
        Let $X$ be the number of failed connections in one hour. Since the average is 4 per hour, $X \sim Po(4)$ [7].
        \begin{enumerate}
            \item[\textbf{i}] 2 users fail to connect on their first attempt ($P(X=2)$):
            \begin{align*}
            P(X=2) &= \frac{e^{-4} 4^2}{2!} \\
            &= \frac{0.018316 \times 16}{2} \approx \textbf{0.1465} \quad \text{(4 d.p.)} [7].
            \end{align*}
            \item[\textbf{ii}] more than 6 users fail to connect on their first attempt ($P(X > 6)$):
            \begin{align*}
            P(X > 6) &= 1 - P(X \le 6) \\
            &= 1 - 0.8893 = \textbf{0.1107} \quad \text{(4 d.p.)} [7].
            \end{align*}
        \end{enumerate}
        \item[\textbf{c}] Find the probability that in a randomly chosen 90-minute period:
        A 90-minute period is 1.5 hours. The new average rate ($\lambda$) for this period is $4 \times 1.5 = 6$.
        Let $Y$ be the number of failed connections in 90 minutes. $Y \sim Po(6)$ [7].
        \begin{enumerate}
            \item[\textbf{i}] 5 users fail to connect on their first attempt ($P(Y=5)$):
            \begin{align*}
            P(Y=5) &= \frac{e^{-6} 6^5}{5!} \approx \textbf{0.1606} \quad \text{(4 d.p.)} [7].
            \end{align*}
            \item[\textbf{ii}] fewer than 7 users fail to connect on their first attempt ($P(Y < 7)$):
            This means $P(Y \le 6)$.
            \begin{align*}
            P(Y < 7) &= P(Y \le 6) \approx \textbf{0.6063} \quad \text{(4 d.p.)} [7].
            \end{align*}
        \end{enumerate}
    \end{enumerate}
\end{itemize}

\section*{3. Adding Poisson Distributions (Additive Property)}

\begin{itemize}
    \item \textbf{Idea}: This property states that if $X$ and $Y$ are two \textbf{independent} random variables following Poisson distributions with parameters $\lambda$ and $\mu$ respectively (i.e., $X \sim Po(\lambda)$ and $Y \sim Po(\mu)$), then their sum $X+Y$ also follows a Poisson distribution with parameter $\lambda+\mu$ (i.e., $X+Y \sim Po(\lambda+\mu)$) [8]. This is applicable when both distributions model events occurring in the same interval of time or space [8].

    \item \textbf{Solved Example (from Example 6, page 26)} [8]:
    Given $X \sim Po(3.6)$ and $Y \sim Po(4.4)$.
    \begin{enumerate}
        \item[\textbf{a}] Find $P(X+Y=7)$:
        Since $X$ and $Y$ are independent Poisson variables, their sum $X+Y$ follows a Poisson distribution with parameter $\lambda = 3.6 + 4.4 = 8$. So, $X+Y \sim Po(8)$ [8].
        Let $Z = X+Y$. We need $P(Z=7)$.
        \begin{align*}
        P(Z=7) &= \frac{e^{-8} 8^7}{7!} \approx \textbf{0.1396} \quad \text{(4 d.p.)} [8].
        \end{align*}
        \item[\textbf{b}] Find $P(X+Y \le 5)$:
        Using a cumulative distribution table or calculator for $Z \sim Po(8)$,
        \begin{align*}
        P(Z \le 5) &\approx \textbf{0.1912} \quad \text{(4 d.p.)} [8].
        \end{align*}
    \end{enumerate}
\end{itemize}

\section*{4. Mean and Variance of a Poisson Distribution}

\begin{itemize}
    \item \textbf{Idea}: A key characteristic of the Poisson distribution is that its mean (expected value) is equal to its variance [9]. For a random variable $X \sim Po(\lambda)$:
    \begin{itemize}
        \item Mean $E(X) = \lambda$ [9].
        \item Variance $Var(X) = \lambda$ [9].
        \item The standard deviation is $\sqrt{\lambda}$.
    \end{itemize}

    \item \textbf{Solved Example (from Example 8, page 29)} [9, 10]:
    A botanist counts the number of daisies, $x$, in each of 80 randomly selected squares. Summarized results: $\Sigma x = 295$, $\Sigma x^2 = 1386$.
    \begin{enumerate}
        \item[\textbf{a}] Calculate the mean and the variance of the number of daisies per square. Give your answers to 2 decimal places.
        \begin{itemize}
            \item \textbf{Mean ($E(X)$)}:
            \begin{align*}
            E(X) &= \frac{\Sigma x}{n} = \frac{295}{80} = 3.6875 \approx \textbf{3.69} \quad \text{(2 d.p.)} [10].
            \end{align*}
            \item \textbf{Variance ($Var(X)$)}:
            \begin{align*}
            Var(X) &= \frac{\Sigma x^2}{n} - (\text{mean})^2 \\
            &= \frac{1386}{80} - (3.6875)^2 \\
            &= 17.325 - 13.59140625 \approx 3.73359375 \approx \textbf{3.73} \quad \text{(2 d.p.)} [10].
            \end{align*}
        \end{itemize}
        \item[\textbf{b}] Explain how the answers from part a support the choice of a Poisson distribution as a model.
        The mean ($\approx 3.69$) and the variance ($\approx 3.73$) are \textbf{approximately equal} [10]. This equality (or near-equality) between the mean and variance is a fundamental property of the Poisson distribution, making it a suitable model [9].
        \item[\textbf{c}] Using a suitable value for $\lambda$, estimate the probability that exactly 3 daisies will be found in a randomly selected square.
        Using the estimated mean as the parameter for the Poisson distribution, $\lambda = 3.7$ [10].
        Let $X \sim Po(3.7)$. We need to find $P(X=3)$.
        Using tables or a calculator for $Po(3.7)$,
        \begin{align*}
        P(X=3) &\approx \textbf{0.2087} \quad \text{(4 d.p.)} [10].
        \end{align*}
    \end{enumerate}
\end{itemize}

\vspace{0.5cm}
Chapter 2 concludes with a Chapter Review section, offering further practice problems related to these concepts [11].

\end{document}
