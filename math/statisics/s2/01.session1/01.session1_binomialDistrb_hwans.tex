\documentclass{article}
\usepackage{amsmath}
\usepackage{amssymb}
\usepackage{enumerate}

\begin{document}

\title{Binomial Distribution Solutions}
\author{}
\date{}
\maketitle

\section*{Solutions}

\begin{enumerate}
    \item \textbf{Basic Formula Usage}
    The binomial probability formula is given by $P(X=k) = \binom{n}{k} p^k (1-p)^{n-k}$, where $n$ is the number of trials, $k$ is the number of successes, and $p$ is the probability of success on a single trial.
    \begin{enumerate}[a)]
        \item Here, $n=5$, $k=3$, and $p=0.6$.
        $P(X=3) = \binom{5}{3} (0.6)^3 (1-0.6)^{5-3} = 10 \times (0.6)^3 \times (0.4)^2 = 10 \times 0.216 \times 0.16 = 0.3456$.
        \item Here, $n=10$, $k=7$, and $p=1/4 = 0.25$.
        $P(X=7) = \binom{10}{7} (0.25)^7 (1-0.25)^{10-7} = 120 \times (0.25)^7 \times (0.75)^3 \approx 120 \times 0.000061 \times 0.421875 \approx 0.00309$.
        \item Here, $n=8$, $k=1$, and $p=0.02$.
        $P(X=1) = \binom{8}{1} (0.02)^1 (1-0.02)^{8-1} = 8 \times 0.02 \times (0.98)^7 \approx 0.16 \times 0.8681 \approx 0.1389$.
    \end{enumerate}

    \item \textbf{Cumulative Probabilities}
    \begin{enumerate}[a)]
        \item We need to find $P(X \le 2) = P(X=0) + P(X=1) + P(X=2)$. Here, $n=6$ and $p=0.3$.
        $P(X=0) = \binom{6}{0} (0.3)^0 (0.7)^6 = 1 \times 1 \times 0.117649 = 0.117649$.
        $P(X=1) = \binom{6}{1} (0.3)^1 (0.7)^5 = 6 \times 0.3 \times 0.16807 = 0.302526$.
        $P(X=2) = \binom{6}{2} (0.3)^2 (0.7)^4 = 15 \times 0.09 \times 0.2401 = 0.324135$.
        $P(X \le 2) = 0.117649 + 0.302526 + 0.324135 = 0.74431$.
        \item We need to find $P(X \ge 10) = P(X=10) + P(X=11) + P(X=12)$. Here, $n=12$ and $p=0.75$.
        $P(X=10) = \binom{12}{10} (0.75)^{10} (0.25)^2 = 66 \times 0.0563 \times 0.0625 \approx 0.2323$.
        $P(X=11) = \binom{12}{11} (0.75)^{11} (0.25)^1 = 12 \times 0.0423 \times 0.25 \approx 0.1269$.
        $P(X=12) = \binom{12}{12} (0.75)^{12} (0.25)^0 = 1 \times 0.0317 \times 1 \approx 0.0317$.
        $P(X \ge 10) = 0.2323 + 0.1269 + 0.0317 \approx 0.3909$.
        \item We need to find $P(X < 3) = P(X=0) + P(X=1) + P(X=2)$. Here, $n=9$ and $p=0.15$.
        $P(X=0) = \binom{9}{0} (0.15)^0 (0.85)^9 \approx 0.2316$.
        $P(X=1) = \binom{9}{1} (0.15)^1 (0.85)^8 \approx 0.3679$.
        $P(X=2) = \binom{9}{2} (0.15)^2 (0.85)^7 \approx 0.2597$.
        $P(X < 3) = 0.2316 + 0.3679 + 0.2597 = 0.8592$.
    \end{enumerate}

    \item \textbf{Mean and Variance}
    The mean of a binomial distribution is $E(X) = \mu = np$ and the variance is $Var(X) = \sigma^2 = np(1-p)$. The standard deviation is $\sigma = \sqrt{np(1-p)}$.
    \begin{enumerate}[a)]
        \item Here, $n=50$ and $p=0.8$.
        Mean: $\mu = 50 \times 0.8 = 40$.
        Variance: $\sigma^2 = 50 \times 0.8 \times (1-0.8) = 50 \times 0.8 \times 0.2 = 8$.
        \item Here, $n=150$ and $p=0.4$.
        Mean: $\mu = 150 \times 0.4 = 60$.
        Standard Deviation: $\sigma = \sqrt{150 \times 0.4 \times (1-0.4)} = \sqrt{150 \times 0.4 \times 0.6} = \sqrt{36} = 6$.
        \item Here, $n=1000$ and $p=0.01$.
        Mean: $\mu = 1000 \times 0.01 = 10$.
        Variance: $\sigma^2 = 1000 \times 0.01 \times (1-0.01) = 1000 \times 0.01 \times 0.99 = 9.9$.
    \end{enumerate}
\end{enumerate}
\end{document}
