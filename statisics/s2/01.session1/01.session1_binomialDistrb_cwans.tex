\documentclass[12pt]{article}
\usepackage{amsmath}
\usepackage{amssymb}
\usepackage{geometry}
\geometry{a4paper, margin=1in}
\usepackage{fancyhdr}
\pagestyle{fancy}
\fancyhead[L]{Pearson Edexcel IAL Statistics 2}
\fancyhead[R]{Chapter 1: Binomial Distributions}
\fancyfoot[C]{\thepage}
\renewcommand{\headrulewidth}{0.4pt}
\renewcommand{\footrulewidth}{0.4pt}

\begin{document}

\centerline{\Large\textbf{Answers: Chapter 1 - Binomial Distributions}}
\centerline{\textbf{Pearson Edexcel International A Level Statistics 2}}
\vspace{0.5cm}

\textbf{Instructions:} Answer all questions, showing clear working where appropriate. You may use a calculator, and refer to binomial cumulative distribution tables where applicable.

\vspace{0.5cm}

\begin{enumerate}
    \item \textbf{Understanding the Binomial Distribution Definition and Conditions}
    A random variable $X$ is said to follow a binomial distribution, denoted as $X \sim B(n, p)$.
    \begin{enumerate}
        \item The four conditions that a random variable must satisfy for it to be modeled by a binomial distribution are:
        \begin{itemize}
            \item There is a \textbf{fixed number of trials}, $n$.
            \item Each trial has only \textbf{two possible outcomes} (success or failure).
            \item The probability of success, $p$, is \textbf{constant} for each trial.
            \item The trials are \textbf{independent} of each other.
        \end{itemize}

        \item The number of defective electronic components can be modeled by a binomial distribution because there is a fixed number of trials ($n=20$ components), each trial has two outcomes (defective or not defective), and the probability of a defective component is assumed to be constant for each component. The random selection implies that the trials are independent.
    \end{enumerate}

    \item \textbf{Calculating Probabilities using the Binomial Probability Formula}
    Let $X$ be a random variable such that $X \sim B(10, 0.3)$.
    \begin{enumerate}
        \item The binomial probability formula is $P(X=x) = \binom{n}{x} p^x (1-p)^{n-x}$.
        \begin{align*}
        P(X=4) &= \binom{10}{4} (0.3)^4 (1-0.3)^{10-4} \\
        &= 210 \times (0.3)^4 \times (0.7)^6 \\
        &= 210 \times 0.0081 \times 0.117649 \\
        &\approx 0.2001
        \end{align*}

        \item For a multiple-choice test with 8 questions and 4 options per question, the number of trials is $n=8$ and the probability of a correct guess is $p = \frac{1}{4} = 0.25$. Let $Y \sim B(8, 0.25)$.
        \begin{align*}
        P(Y=3) &= \binom{8}{3} (0.25)^3 (1-0.25)^{8-3} \\
        &= 56 \times (0.25)^3 \times (0.75)^5 \\
        &= 56 \times 0.015625 \times 0.2373 \\
        &\approx 0.2076
        \end{align*}
    \end{enumerate}

    \item \textbf{Working with Cumulative Probabilities}
    A biased coin is tossed 15 times, with $p=0.6$ for a head. Let $H \sim B(15, 0.6)$.
    \begin{enumerate}
        \item The probability of obtaining no more than 7 heads is $P(H \le 7)$. Using a binomial cumulative distribution table or calculator, we find:
        $$ P(H \le 7) \approx 0.2131 $$

        \item The probability of obtaining at least 10 heads is $P(H \ge 10)$. We use the identity $P(H \ge 10) = 1 - P(H \le 9)$.
        From the cumulative distribution table or calculator, $P(H \le 9) \approx 0.8752$.
        $$ P(H \ge 10) = 1 - P(H \le 9) = 1 - 0.8752 = 0.1248 $$
    \end{enumerate}

    \item \textbf{Calculating Mean and Variance of a Binomial Distribution}
    For a random variable $X \sim B(25, 0.4)$.
    \begin{enumerate}
        \item The mean (expected value) is $E(X) = np$.
        $$ E(X) = 25 \times 0.4 = 10 $$

        \item The variance is $Var(X) = np(1-p)$.
        $$ Var(X) = 25 \times 0.4 \times (1-0.4) = 25 \times 0.4 \times 0.6 = 6 $$
        The standard deviation is the square root of the variance.
        $$ \text{Standard Deviation} = \sqrt{Var(X)} = \sqrt{6} \approx 2.449 $$
    \end{enumerate}

    \item \textbf{Problem-Solving and Real-World Applications}
    A manufacturer states that $15\%$ of its light bulbs are faulty. A quality control inspector randomly selects 12 light bulbs.
    \begin{enumerate}
        \item The distribution of the number of faulty light bulbs, $F$, is a binomial distribution with parameters $n=12$ and $p=0.15$.
        $$ F \sim B(12, 0.15) $$

        \item We need to find the probability that there are exactly 2 faulty light bulbs, $P(F=2)$.
        \begin{align*}
        P(F=2) &= \binom{12}{2} (0.15)^2 (0.85)^{10} \\
        &= 66 \times 0.0225 \times 0.19687 \\
        &\approx 0.2924
        \end{align*}
    \end{enumerate}

    \item \textbf{Finding Unknown Parameters}
    A random variable $X$ follows a binomial distribution $B(n, p)$.
    \begin{enumerate}
        \item Given that $E(X) = 6$ and $n=20$:
        $$ E(X) = np \implies 6 = 20p \implies p = \frac{6}{20} = 0.3 $$

        \item Given that $E(X) = 4.8$ and $Var(X) = 2.88$:
        $$ E(X) = np = 4.8 \quad \text{(1)} $$
        $$ Var(X) = np(1-p) = 2.88 \quad \text{(2)} $$
        Substitute (1) into (2):
        $$ 4.8(1-p) = 2.88 $$
        $$ 1-p = \frac{2.88}{4.8} = 0.6 $$
        $$ p = 1 - 0.6 = 0.4 $$
        Substitute $p=0.4$ back into (1):
        $$ n(0.4) = 4.8 \implies n = \frac{4.8}{0.4} = 12 $$
        The values are $n=12$ and $p=0.4$.
    \end{enumerate}

    \item \textbf{Comprehensive Problem-Solving / Justification (Exam-style)}
    A market research firm conducts a telephone survey, with the probability of a completed survey being $p=0.2$. A researcher makes 15 calls.
    \begin{enumerate}
        \item Let $S$ be the number of completed surveys. Two assumptions needed to model $S$ using a binomial distribution are:
        \begin{itemize}
            \item **Independence of trials:** Each phone call is independent of the others. This is a reasonable assumption as the result of one call is unlikely to influence the outcome of another.
            \item **Constant probability of success:** The probability of a person answering and completing the survey is constant for each call. This is reasonable if the calls are made to a random sample under similar conditions.
        \end{itemize}

        \item We need to find $P(3 \le S < 6)$, which is equivalent to $P(S \le 5) - P(S \le 2)$. For $S \sim B(15, 0.2)$:
        Using a cumulative distribution table or a calculator:
        $$ P(S \le 5) \approx 0.9389 $$
        $$ P(S \le 2) \approx 0.6482 $$
        $$ P(3 \le S < 6) = 0.9389 - 0.6482 = 0.2907 $$
    \end{enumerate}
\end{enumerate}

\end{document}
