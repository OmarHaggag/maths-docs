\documentclass{article}
\usepackage{amsmath}

\begin{document}

\section*{Problem 3}

Define \( S_n = \sum_{k=1}^n \left( 3k^2 - 3k + 1 \right) \).

We are tasked to compute \( S_1, S_2, S_3, S_4 \) and conjecture a formula for \( S_n \).

\subsection*{Step 1: Expand the sum}

\[
S_n = \sum_{k=1}^n \left( 3k^2 - 3k + 1 \right)
= 3 \sum_{k=1}^n k^2 - 3 \sum_{k=1}^n k + \sum_{k=1}^n 1.
\]

Using known summation formulas:
\begin{align*}
\sum_{k=1}^n k^2 & = \frac{n(n+1)(2n+1)}{6}, \\
\sum_{k=1}^n k & = \frac{n(n+1)}{2}, \\
\sum_{k=1}^n 1 & = n.
\end{align*}

Substitute into the equation:
\[
S_n = 3 \cdot \frac{n(n+1)(2n+1)}{6} - 3 \cdot \frac{n(n+1)}{2} + n.
\]

Simplify:
\[
S_n = \frac{n(n+1)(2n+1)}{2} - \frac{3n(n+1)}{2} + n.
\]

\[
S_n = \frac{n(n+1)}{2} \left( 2n+1 - 3 \right) + n.
\]

\[
S_n = \frac{n(n+1)(2n-2)}{2} + n
= n(n+1)(n-1) + n.
\]

\subsection*{Step 2: Evaluate \( S_1, S_2, S_3, S_4 \)}

1. \( S_1 \):
\[
S_1 = 1(1+1)(1-1) + 1 = 0 + 1 = 1.
\]

2. \( S_2 \):
\[
S_2 = 2(2+1)(2-1) + 2 = 2(3)(1) + 2 = 6 + 2 = 8.
\]

3. \( S_3 \):
\[
S_3 = 3(3+1)(3-1) + 3 = 3(4)(2) + 3 = 24 + 3 = 27.
\]

4. \( S_4 \):
\[
S_4 = 4(4+1)(4-1) + 4 = 4(5)(3) + 4 = 60 + 4 = 64.
\]

\subsection*{Step 3: Conjecture Formula for \( S_n \)}

From the results:
\[
S_n = 1, 8, 27, 64.
\]

These correspond to the cubes of integers:
\[
S_n = n^3.
\]

Thus, the formula is:
\[
S_n = n^3.
\]

\end{document}
